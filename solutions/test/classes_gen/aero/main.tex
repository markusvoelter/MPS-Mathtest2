\documentclass{article}
\usepackage[latin1]{inputenc}

\begin{document}
\section{Aircraft}
\begin{tabular}{l l l | l}
Symbol & Unit & Java Type & Description \\ \hline
$v$  &  m/s  &  double  &  current aircraft speed \\ 
$A$  &  m m  &  double  &  cross area of the wing \\ 
$c_a$  &  -  &  double  &  Auftriebsbeiwert \\ 
$c_w$  &  -  &  double  &  Widerstandsbeiwert \\ 
\end{tabular}
\section{Environment}
\begin{tabular}{l l l | l}
Symbol & Unit & Java Type & Description \\ \hline
$rho$  &  kg/m m m  &  double  &  The density of the air \\ 
\end{tabular}
\section{Fundamental Stuff}
lkfjdslkjfsld lsdkjf lsdkjf sdur weoiu sdofiud gkfjg dflkjg df $p_dyn$ is calculated from the $c_w$ current air density $rho$ and the square of the flight speed $v$ dlkfjldskjfls jsldaf ljfds lf jsaldjfldsajf $c_a$ ldsa $c_a$ 


\vspace{3 mm}
$p_dyn$ \hspace{1 mm} = \hspace{1 mm} $rho\hspace{1 mm}v^{2}$ 


\vspace{3 mm}


\vspace{3 mm}
$someFct$ \hspace{1 mm} = \hspace{1 mm} $\frac{1}{2}\hspace{1 mm}17$ 


\vspace{3 mm}
\section{Stuff on the Wings}
Aus dem Staudruck $p_dyn$ lässt sich dann der aktuelle Auftrieb $F_A$ berechnen; die Form wird duch den Auftriebsbeiwert $c_a$ beschrieben  und die Fläche durch  $A$ 


\vspace{3 mm}
$F_A$ \hspace{1 mm} = \hspace{1 mm} $p_dyn\hspace{1 mm}A\hspace{1 mm}c_a\hspace{1 mm}A$ 


\vspace{3 mm}
Auch der Widerstand $F_W$ berechnet sich entsprechend mit Hilfe des Beiwertes $c_w$ : 


\vspace{3 mm}
$F_W$ \hspace{1 mm} = \hspace{1 mm} $p_dyn\hspace{1 mm}A\hspace{1 mm}c_w$ 


\vspace{3 mm}
\end{document}
